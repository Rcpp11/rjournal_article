% !TeX root = RJwrapper.tex
\title{R and C++11}
\author{by Romain François}

\maketitle

\abstract{}

\section{Introduction}

Extending R with compiled code is a great way to achieve good performance. 
Using C++ to rewrite critical parts has become a popular approach
in R package development. 

With the release of long awaited C++11 standard~\citep{Cpp11}, 
the C++ Standards Commitee has changed the meaning of C++, making it 
a language that is nicer to use and teach. 

This article has two goals, 
showing a few examples of C++11 language features that might be 
relevant to R package development, and introducing
Rcpp11, a complete C++11 redesign of \CRANpkg{Rcpp}. 

\section{Sightseeing tour of C++11}

This section presents a few features of C++11 which might be 
relevant to R package developpers using C++. For a complete 
reference of C++11, see~\citep{Stroustrup2013}. 

\subsection{auto}

An example with sugar, where the return type is complicated to express
by hand. Also mention \texttt{decltype}

\subsection{Range based for loops}

They look like R loops. 

\subsection{Lambda functions}

Feels like using sapply ...

\subsection{Concurrency and threads}

Machines with multiple cores or even multipe 
processors are becoming increasingly available. Best performance 
is nowadays achieved by writing multithreaded code. 

In previous versions of C++, this involves using external libraries 
such as \texttt{Boost.Threads}~\citep{KempfBoostThreads} 
or fall back to using platform specific 
implementations of threads, for example POSIX threads. 

\subsubsection{Example}

As a simple example, let's consider the problem of counting the number
of positive values from a numeric vector. In R, we would probably express this
with this function: 

\begin{example}
count_positive_R <- function(x){
  sum(x > 0)
}
\end{example}

In C++, we might express the algorithm using the \code{count\_if} template
function from the STL, combined with a lambda function:

\begin{example}
typedef NumericVector::iterator Iterator ;

struct count_positive {
  int operator()(Iterator first, Iterator last){
    return std::count_if( first, last, 
      [](double x){ return x > 0 ; } 
    );    
  }
} ;

// [[Rcpp::export]]
int count_positive_serial(NumericVector data){
  return count_positive()(data.begin(), data.end());
}
\end{example}

The advantage of delegating the work to the range-based callable 
\code{count\_positive} struct is that we can easily 
parallelize the code. We simply split the range into a set of ranges, 
each of which to be processed in a thread. 

\begin{example}
#include <thread>
#include <future>

typedef NumericVector::iterator Iterator ;
typedef std::packaged_task<int(Iterator,Iterator)> Task ;

// [[Rcpp::export]]
int count_positive_threaded(NumericVector data, int nthreads){
  int n = data.size() ;
  int chunk_size = n / nthreads ; 
  
  std::vector<std::future<int>> futures(nthreads-1) ;
  std::vector<std::thread> threads(nthreads-1) ;
  
  Iterator it = data.begin() ;
  for( int i=0; i<nthreads-1; i++){
    count_positive counter ;        
    Task task(counter) ;
    futures[i] = task.get_future();
    threads[i] = std::thread( std::move(task), it, it + chunk_size ) ;
    it += chunk_size ;
  }
  
  int result = count_positive()(it, data.end()); 
  
  for( int i=0; i<nthreads-1; i++){
    threads[i].join() ;
  }
  for( int i=0; i<nthreads-1; i++){
    result += futures[i].get() ;  
  }
                            
  return result ;
}
\end{example}

For example, with \code{nthreads=4} the three first quarters of the data
are processed in their own thread and the last quarter is processed by the 
main thread. Then the main thread joins the 3 worker threads and collect
the results. This example used \code{thread}, \code{future} and
\code{packaged\_task} from the C++ Standard Library. 

\begin{table}
\centering
\begin{tabular}{lrrrrr}
\toprule 
$n$ & $10^5$ & $10^6$ & $10^7$ & $10^8$ & $10^9$ \\ 
\midrule          
R         & 0.43 & 3.32  & 34.42 & 467.03 & 4\,977.05 \\
serial    & 0.09 & 0.91  &  9.25 &  97.59 &    854.21 \\
\hspace{1.5cm} & 
\hspace{1.5cm} & 
\hspace{1.5cm} &
\hspace{1.5cm} & 
\hspace{1.5cm} & 
\hspace{1.5cm} \\
2 threads & 0.12 & 0.54  &  4.91 &  49.37 &    496.15 \\
4 threads & 0.15 & 0.35  &  2.76 &  37.47 &    376.31 \\
8 threads & 0.18 & 0.36  &  2.57 &  37.70 &    371.98 \\
\bottomrule
\end{tabular}
\caption{\label{table:count}Run times (ms). Median run times between 10 runs.\\ 
{\footnotesize Timings performed on a Mac Book Pro with a 2.3 GHz Intel Core i7 processor (4 cores), 16Go of RAM. 
Compiled with clang++ version 3.3 with the flags : \code{-g -O3}. R version 3.0.2 
as distributed by the CRAN binary for OSX. }}
\end{table}

Table~\ref{table:count} presents some benchmarks with various 
data sizes (from $10^5$ to $ 10^9$). 

What to say: 

threads have their cost, with too small data sizes, it is actually more
expensive to use threads. 

R code vectorized but slow, has to allocate memory for the intermediate 
logical vector. 

We could split the work differently between threads. 

\section{Rcpp11}

\subsection{Motivation}

Complete redesign of Rcpp
Header only. Assumes C++11.
Cleaner (got rid of some ghosts). 

\subsection{Future development}

Threaded sugar. 

\section{Beyond C++11}

\bibliography{Francois}

\address{Romain François\\
    R Enthusiasts\\
    1 place de l'égalité. 42400 Saint Chamond\\
    FRANCE }
\email{romain@r-enthusiasts.com}
    

